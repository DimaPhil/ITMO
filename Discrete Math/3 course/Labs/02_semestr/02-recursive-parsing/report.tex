\documentclass[11pt,a4paper,oneside]{article}

\usepackage[english,russian]{babel}
\usepackage[T2A]{fontenc}
\usepackage[utf8]{inputenc}
\usepackage{graphicx}
\usepackage{expdlist}
\usepackage{mfpic}
\usepackage{amsmath}
\usepackage{amssymb}
\usepackage{comment}
%\usepackage{listings}
\usepackage{epigraph}
\usepackage{url}
\usepackage{ulem}

\DeclareMathOperator{\nott}{not}

\begin{document}

\renewcommand{\t}[1]{\mbox{\texttt{#1}}}
\newcommand{\s}[1]{\mbox{``\t{#1}''}}
\newcommand{\eps}{\varepsilon}
\renewcommand{\phi}{\varphi}
\newcommand{\plainhat}{{\char 94}}

\newcommand{\Z}{\mathbb{Z}}
\newcommand{\w}[1]{``\t{#1}''}

\binoppenalty=10000
\relpenalty=10000

\epigraph{Филиппов Дмитрий, M3339}

\begin{LARGE} \textbf{Лабораторная работа №2. Ручное построение нисходящих синтаксических анализаторов} \end{LARGE}
\newline

\textbf{1. Разработка грамматики}
\newline                    
\newline

Построим грамматику.

\begin{itemize}
  \item $E \rightarrow \mbox{C | E or C}$
  \item $C \rightarrow \mbox{X | C and X}$
  \item $X \rightarrow \mbox{N | X xor N}$
  \item $N \rightarrow \mbox{[a-z] | not N | (E)}$
\end{itemize}

\begin{tabular}{|c||r|rrr||}
\hline
 & Нетерминал & Описание \\
\hline
\hline
 & E (Expression) & Логическое выражение \\
 & C (Conjuction) & Конъюнкция \\
 & X (Xor) & Ксор \\
 & N (Negate) & Отрицание \\
\hline
\end{tabular}
\newline

В грамматике есть левая рекурсия. Устраним ее. Получится грамматика:

\begin{itemize}
  \item $E \rightarrow \mbox{CE' | C}$
  \item $E' \rightarrow \mbox{or CE' | or C}$
  \item $C \rightarrow \mbox{XC' | X}$
  \item $C' \rightarrow \mbox{and XC' | and X}$
  \item $X \rightarrow \mbox{NX' | N}$
  \item $X' \rightarrow \mbox{xor NX' | xor N}$
  \item $N \rightarrow \mbox{[a-z] | not N | (E)}$
\end{itemize}

А потом упростим:

\begin{itemize}
  \item $E \rightarrow CE'$
  \item $E' \rightarrow \mbox{or CE' | $\eps$}$
  \item $C \rightarrow XC'$
  \item $C' \rightarrow \mbox{and XC' | $\eps$}$
  \item $X \rightarrow NX'$
  \item $X' \rightarrow \mbox{xor NX' | $\eps$}$
  \item $N \rightarrow \mbox{[a-z] | not N | (E)}$
\end{itemize}

\begin{tabular}{|c||r|rrr||}
\hline
 & Нетерминал & Описание \\
\hline
\hline
 & E  & Логическое выражение \\
 & E' & Продолжение логического выражения \\
 & C  & Конъюнкция \\
 & C' & Продолжение конъюнкции \\
 & X  & Ксор \\
 & X' & Продолжение ксора \\
 & N  & Отрицание \\
\hline
\end{tabular}
\newline

\textbf{2. Построение лексического анализатора}
\newline                    
\newline

В грамматике терминалы~--- $[a-z]$, $or$, $xor$, $and$, $not$.

Построим лексический анализатор. Заведем класс $Token$ для хранения терминалов.

$public$ $enum$ $Token$ $\{$

    $LPAREN,$ $RPAREN,$ $VARIABLE,$ $AND,$ $OR,$ $XOR,$ $NOT,$ $END$

$\}$

\begin{tabular}{|c||r|rrr|}
\hline
 & Терминал & Токен \\
\hline
\hline
 & ( & LPAREN  \\
 & ) & RPAREN \\
 & a-z & VARIABLE \\
 & not & NOT \\
 & and & AND \\
 & or & OR \\
 & xor & XOR \\
 & \$ & END \\
\hline
\end{tabular}
\newline

\textbf{3. Построение синтаксического анализатора}
\newline                    
\newline

Построим множества $FIRST$ и $FOLLOW$ для нетерминалов нашей грамматики.

\begin{tabular}{|c||r|r|rrr||}
\hline
 & Нетерминал & FIRST & FOLLOW \\
\hline
\hline
 & E  & (, a-z, not  & ), \$ \\
 & E' & or, $\eps$   & ), \$ \\
 & C  & (, a-z, not  & or, ), \$ \\
 & C' & and, $\eps$  & or, ), \$ \\
 & X  & (, a-z, not  & and, or, ), \$ \\
 & X' & xor, $\eps$  & and, or, ), \$ \\
 & N  & (, a-z, not, & xor, and, or, ), \$ \\
\hline
\end{tabular}
\newline

Тесты:

\begin{itemize}
 \item $a, m, z$~--- Тесты на правило $N$ $\rightarrow$ $[a-z]$
 \item $\mbox{a and b}$~--- Тест на примитивную операцию
 \item $\mbox{a xor b}$~--- Тест на примитивную операцию
 \item $\mbox{a or b}$~--- Тест на примитивную операцию
 \item $not$ $a$~--- Тест на примитивную операцию
 \item $(a)$~--- Тест на правило $(E)$
 \item $\mbox{(((((a and b)))))}$~--- Тест с большим количеством вложенных скобок
 \item $\mbox{(a and (b and (c or (not d xor not c))) or not c)}$~--- Большой тест
 \item $\mbox{not not not not not b}$~--- Множественное применение not
 \item $($~--- Тест на неккоректную скобочную последовательность
 \item $)$~--- Тест на неккоректную скобочную последовательность
 \item $\mbox{(a an b)}$~--- Тест на неккоректную операцию
 \item $\mbox{(a and (b xor ))}$~--- Тест на недостающую переменную
\end{itemize}

\end{document}