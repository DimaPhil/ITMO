\documentclass[11pt,a4paper,oneside]{article}

\usepackage[english,russian]{babel}
\usepackage[T2A]{fontenc}
\usepackage[utf8]{inputenc}
\usepackage{graphicx}
\usepackage{expdlist}
\usepackage{mfpic}
\usepackage{amsmath}
\usepackage{amssymb}
\usepackage{comment}
%\usepackage{listings}
\usepackage{epigraph}
\usepackage{url}
\usepackage{ulem}

\DeclareMathOperator{\nott}{not}

\begin{document}

\renewcommand{\t}[1]{\mbox{\texttt{#1}}}
\newcommand{\s}[1]{\mbox{``\t{#1}''}}
\newcommand{\eps}{\varepsilon}
\renewcommand{\phi}{\varphi}
\newcommand{\plainhat}{{\char 94}}

\newcommand{\Z}{\mathbb{Z}}
\newcommand{\w}[1]{``\t{#1}''}

\binoppenalty=10000
\relpenalty=10000

\epigraph{Филиппов Дмитрий, M3339}

\begin{LARGE} \textbf{Лабораторная работа №2. Ручное построение нисходящих синтаксических анализаторов} \end{LARGE}
\newline

\textbf{1. Разработка грамматики}
\newline                    
\newline

Построим грамматику.

\begin{itemize}
  \item $S \rightarrow not S$
  \item $S \rightarrow TS'$
  \item $S' \rightarrow \mbox{or TS'}|\eps$
  \item $T \rightarrow FT'$
  \item $T' \rightarrow \mbox{xor FT'}|\eps$
  \item $F \rightarrow GP'$
  \item $P' \rightarrow \mbox{and GP'}|\eps$
  \item $G \rightarrow (S)|[a-z]$
\end{itemize}

\begin{tabular}{|c||r|rrr||}
\hline
 & Нетерминал & Описание \\
\hline
\hline
 & S & Логическое выражение \\
 & T & Логическое выражение, содержащее только $and$, $xor$ \\
 & S' & Продолжение логического выражения \\
\hline
\end{tabular}
\newline

В грамматике есть левая рекурсия. Устраним ее. Получится грамматика:

\begin{itemize}
  \item $S' \rightarrow OSS'|OS$
  \item $S \rightarrow (S)S'|not$ $SS'|[a-z]S'$
  \item $S \rightarrow (S)|[a-z]|not$ $S$
  \item $O \rightarrow and|or|xor$
\end{itemize}

А потом упростим:

\begin{itemize}
  \item $S' \rightarrow OSS'|\eps$
  \item $S \rightarrow (S)S'|not$ $SS'|[a-z]S'$
	\item $O \rightarrow and|or|xor$
\end{itemize}

\begin{tabular}{|c||r|rrr||}
\hline
 & Нетерминал & Описание \\
\hline
\hline
 & S & логическое выражение  \\
 & S' & продолжение логического выражения \\
 & O & Логическая операция \\
\hline
\end{tabular}
\newline

\textbf{2. Построение лексического анализатора}
\newline                    
\newline

В грамматике терминалы~--- $[a-z]$, $or$, $xor$, $and$.

Построим лексический анализатор. Заведем класс $Token$ для хранения терминалов.

$public$ $enum$ $Token$ $\{$

    $LPAREN,$ $RPAREN,$ $VARIABLE,$ $AND,$ $OR,$ $XOR,$ $NOT,$ $END$

$\}$

\begin{tabular}{|c||r|rrr|}
\hline
 & Терминал & Токен \\
\hline
\hline
 & ( & LPAREN  \\
 & ) & RPAREN \\
 & a-z & VARIABLE \\
 & not & NOT \\
 & and & AND \\
 & or & OR \\
 & xor & XOR \\
 & \$ & END \\
\hline
\end{tabular}
\newline

\textbf{3. Построение синтаксического анализатора}
\newline                    
\newline

Построим множества $FIRST$ и $FOLLOW$ для нетерминалов нашей грамматики.

\begin{tabular}{|c||r|r|rrr||}
\hline
 & Нетерминал & FIRST & FOLLOW \\
\hline
\hline
 & S & (, a-z, not & ), and, or, xor, \$ \\
 & S' & and, or, xor, $\eps$ & ), \$ \\
 & O & and, or, xor & (, a-z, not \\
\hline
\end{tabular}
\newline

Тесты:

\begin{itemize}
 \item $a, m, z$~--- Тесты на правило $S$ $\rightarrow$ $[a-z]S'$
 \item $\mbox{a and b}$~--- Тест на правило $SOS$
 \item $\mbox{a xor b}$~--- Тест на правило $SOS$
 \item $\mbox{a or b}$~--- Тест на правило $SOS$
 \item $not$ $a$~--- Тест на правило $not$ $SS'$
 \item $(a)$~--- Тест на правило $(S)S'$
 \item $\mbox{(a and (b and (c or (not d xor not c))) or not c)}$~--- Большой тест
 \item $\mbox{not not not not not b}$~--- Множественное применение not
 \item $($~--- Тест на неккоректную скобочную последовательность
 \item $)$~--- Тест на неккоректную скобочную последовательность
 \item $\mbox{(a an b)}$~--- Тест на неккоректную операцию
 \item $\mbox{(a and (b xor ))}$~--- Тест на недостающую переменную
\end{itemize}

\end{document}